\documentclass{article}
\usepackage[utf8]{inputenc}
\title{第一回MR}
\author{35714037 糟谷拡輝}
\date{\today}
\begin{document}
\maketitle
\section{MR1}
鼻歌で曲を特定するアルゴリズムを例に挙げます。このアルゴリズムはスマートフォンなどで、ある特定のフレーズだけは知っているが名前のわからない曲の名前を調べるのに使います。このアルゴリズムは、鼻歌で特定のフレーズを入力すると、探している曲を予測して列挙してくれます。
\section{MR2}
minの初期値が0になっているので、毎月の株価が0以下になる場合しかminが新たな値になることがないが、株価がマイナスになることはないと考えられるのでmin=0のままである。しかし、この時mxp=maxとなり求めたい結果とは異なる結果となってしまう。
\section{MR3}
(1)BC(2)ABC(3)AB(4)ABCDF(5)AC
\subsection*{タイトな評価のもの}
(2)C(5)A
\end{document}